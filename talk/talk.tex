% small.tex
\documentclass{beamer}
%include polycode.fmt
\usepackage{xcolor}
\usepackage{media9}
\usepackage{tikz}
\usetikzlibrary{patterns}
\usepackage{pgfplots}
\usepackage[normalem]{ulem}
\usepackage{color, colortbl}
\usepackage{tikz}
\usetikzlibrary{trees}
\usetikzlibrary{shapes}

\definecolor{green}{rgb}{0,1,0}
\usetheme{Antibes}
\useoutertheme[subsection=false]{miniframes}
\usenavigationsymbolstemplate{}
\newsavebox\MBox
\newcommand\Cline[2][red]{{\sbox\MBox{$#2$}%
  \rlap{\usebox\MBox}\color{#1}\rule[-1.2\dp\MBox]{\wd\MBox}{0.5pt}}}
  
\title{Monadic Functional Reactive Programming}
\author{Atze van der Ploeg}
\institute{
Centrum Wiskunde \& Informatica, Amsterdam, The Netherlands}

\newcommand{\dfcode}[1]{\begin{flalign*}\vspace{-0.35cm}#1\vspace{-0.35cm}\end{flalign*}}
\newcommand{\mul}{\!\times\!}
% \date{\today}
\begin{document}

% \AtBeginSection[]
% {
%    \begin{frame}
%        \frametitle{Outline}
%        \tableofcontents[currentsection]
%    \end{frame}
% }


%--- the titlepage frame -------------------------%

\begin{frame}[plain]
\begin{center}
  \scalebox{12}{$\bind$}
\end{center}
\vspace{-0.5cm}
  \titlepage
\end{frame}
\section{Intro}
%format =~ = "\approx"
%format ... = "\dots"
\begin{frame}{What is FRP?}
\begin{block}{Transformational vs. Reactive}
\begin{itemize}
\item |TransformationalProgram =~ Input -> Output|\\
Examples: \begin{itemize}
\item Traditional compiler
\item Compute math expression
\end{itemize}

\item |ReactiveProgram =~ Event -> (Output, ReactiveProgram) |
Examples: \begin{itemize}
\item IDE
\item Spreadsheet
\item Any program with GUI
\end{itemize}
\end{itemize}
\end{block}
\pause
\begin{block}{Functional Reactive Programming(FRP)}
Umbrella term for ways to describe and composing reactive components in a functional way.
\end{block}
\end{frame}

\begin{frame}{Monadic Functional Reactive Programming}
\begin{enumerate}
\item A novel Monadic programmer interface for FRP
\item A novel evaluation mechanism for FRP

\end{enumerate}
\end{frame}

\section{Programmer interface}

\begin{frame}{Reactive computations}

\begin{block}{Reactive computation}
A monadic computation which may require the occurance of external events to continue
\end{block}

\begin{code}
data R ev a =~ Await (ev -> React a) | Done a

instance Monad (R ev) where ...
\end{code}
Example:
\begin{code}
sameClick :: React GUIEv Bool
sameClick = do  p1 <- mouseDown
                p2 <- mouseDown
                return (p1 == p2)

data GUIEv  =  MDown { but :: Int} 
            |  ...
\end{code}
\end{frame}

\begin{frame}{Parallel reactive computation composition}
\begin{block}{Parallel composition operator}
\begin{code}
first :: R ev a -> R ev b -> R ev (R ev a, R ev b)
\end{code}
Runs both reactive computations in parallel until either
completes.
\end{block}

Example: 
\begin{code}
before :: R ev a -> R ev b -> R ev Bool
before a b = fmap cmp (first a b) 
  where  cmp (_, Done b) = False
         cmp (Done a, _) = True

doubler :: React GUIEv ()
doubler = do  rightClick
              r <- rightClick `before` sleep 0:2
              if r then return () else doubler
\end{code}
\end{frame}
%format :|     = ":\!\!|\:"
\begin{frame}{Signal computations}
\begin{block}{Signal computation}
A reactive computation that may also \alert{emit} values
\begin{itemize}
\item Can \alert{end}
\item Describes the entire life time of something.
\item Also a monad $\rightarrow$ composing phases
\item Defined in terms of reactive computations
\end{itemize}

\end{block}
\only<1>{
Example: Modal dialog that asks the users name
\begin{enumerate}
\item Wait for dialog to be needed
\item Emit first form of the dialog
\item Emit new forms of the dialog as the user interacts with the dialog
\item Return name of user
\end{enumerate}
}
\only<2>{
\begin{code} 
newtype S ev f r = ...
instance Monad (S ev f) where ...
waitFor   :: R ev r -> S ev f r
emit      :: f -> S ev f ()
\end{code}
}
\end{frame}

\begin{frame}{Example: Drawing program}
\centering
\begin{tabular}{c c}
\includegraphics[width=0.3\textwidth]{01.png}
&
\includegraphics[width=0.3\textwidth]{02.png}
\end{tabular}
\begin{code}
type Box = (Rect,Color)
define        :: S GUIEv Box Rect
chooseColor   :: Rect -> S GUIEvBox Color
clickOn       :: Rect -> R GUIEv ()
box ::  S GUIEv Box ()
box = do  r <- define
          chooseColor r
          waitFor (clickOn r)
\end{code}
\end{frame}


\begin{frame}{Another Signal computations example}
\begin{code} 
newtype S ev f r = ...
instance Monad (S ev f) where ...
waitFor   :: R ev r -> S ev f r
emit      :: f -> S ev f ()
\end{code}

Example: Cycle trough colors until right click 
\begin{code}
cycleColor :: S GUIEv Color Int
cycleColor = cc colors 1 where
   cc (h : t) i = 
     do  emit h
         r <- waitFor (middleClick `before` rightClick)
         if r then cc t (i + 1) else return i
\end{code}

\end{frame}

\begin{frame}{Dynamic lists}
\begin{center}
\includegraphics[width=0.3\textwidth]{05.png}
\end{center}


\begin{code}
boxes :: S GUIEv [Box] ()
boxes = rList box

rList :: S ef f a -> S ef [f] ()
box   :: S GUIEv Box ()
\end{code}

\begin{itemize}
\item Start a new |box| computation, and when it starts, add it to the list and repeat
\item If a |box| computation ends remove it from the list
\end{itemize}
\end{frame}

\begin{frame}{Programmer interface summary}

\end{frame}

\section{Programmer interface comparison}
%format <<< = "<\!\!\!<\!\!\!<"
%format >>> = ">\!\!\!>\!\!\!>"
%format *** = "*\!\!*\!\!*"
%format -< = "\:-\!\!\!<"
\begin{frame}{Comparison with Arrowized FRP}
Monadic FRP:
\begin{code}
cycleColor :: S GUIEv Color Int
cycleColor = cc colors 1 where
  cc (h:t) i = do
    emit h
    r <- waitFor (middleClick `before` rightClick)
    if r then cc t (i+1) else return i
\end{code}
Arrowized FRP:
\begin{code}
cycleColor :: SF MouseDown (Color, Event Int)
cycleColor = cc colors 1 where
  cc (h : t) i = switch ( proc md -> do
      mc  <- notYet <<< middleClick  -< md
      rc  <- rightClick              -< md
      returnA -< ((h,tag rc i), mc)
    ) (\ _ -> cc t (i+1))
\end{code}

\end{frame}

\begin{frame}{Comparison with Arrowized FRP}
Monadic FRP:
\begin{code}
boxes :: S [Box] ()
boxes = dynList (spawn box)
\end{code}
Arrowized FRP:
\begin{code}
type BoxSF  = SF GUIIn (Box,Event ())

boxes :: SF GUIIn [Box]
boxes = boxes' [] >>> arr (map fst) where
  boxes' i = pSwitchList i
    (newBox *** arr toEv >>> arr choose >>> notYet)
    (\e l -> boxes' (mutateList e l))
choose (a,b) = merge (fmap Left a) (fmap Right b)
toEv l =  let l' = map (isNoEvent . snd) l 
          in if and l' then NoEvent else Event l'
mutate l (Left b)   = b : l
mutate l (Right l') = map fst (filter snd (zip l l'))
\end{code}
\end{frame}

\begin{frame}{Advantages \& disadvantages of Monadic FRP API}
Advantages: 
\begin{itemize}
\item Implicit routing
\item Sequencing phases is easier \& more intuitive.
\item Easier dynamic lists 
\end{itemize}
Disadvantages:
\begin{itemize}
\item Currently no reactive-level recursion
\item Explicit memoization needed for reactive-level sharing
\end{itemize}
\end{frame}

\section{Evaluation model}
\begin{frame}{Evaluation model: context}
Other FRP evaluation models either:
\begin{itemize}
\item Re-evaluate the whole expression after each event
\item Use side-effects to prevent redundant re-evalutions
\end{itemize}

Monadic FRP has the first \emph{purely functional} evaluation model that prevents redundant re-evaluations.

\end{frame}

\begin{frame}{Idea}
We do not know if a reactive computation an event will have any effect on a reactive computation:
\begin{code}
data R ev a  =~  Await (ev -> React a) 
             |   Done a
\end{code}
\pause
Communicate which events a reactive computation is intrested in:
\begin{code}
data R ev a  =~  Await (Requests ev) (ev -> React a) 
             |   Done a
\end{code}
No reevalution: Only call a continuation if it is intrested in the occured event

\end{frame}

\begin{frame}{Effects on basic combinators}

\begin{itemize}
\item |a >>= b| requests the same events as |a|
\item |first a b| requests the union of the requests of |a| and |b| 
\end{itemize}
All other (Signal computation) combinators are expressed in terms of |>>=| and |first|

\end{frame}

\centering
\tikzstyle{bag} = [rectangle,text centered,draw=black,semithick,solid]
\tikzstyle{interpret} = [rectangle, rounded corners,text centered,draw=black,semithick]
 \tikzstyle{every node}=[font=\small]

\begin{frame}{Example evaluation}

\begin{tikzpicture}[grow=down, level distance=60pt]
[-]

\node[bag, name=root] {|first|}
  [sibling distance=4.5cm]
  child {node[bag] {|first|}[sibling distance=2cm]
    child {node[bag] {|mouseMove|}
       edge from parent[<-]
       node[xshift=-15pt,midway,fill=white,color=white,text=black]{$\{\mathit{MouseMove}\}$}
    }
    child {node[bag] {|mouseUp|}
       edge from parent[<-]
       node[xshift=+15pt,midway,fill=white,color=white,text=black]{$\{\mathit{MouseUp}\}$}
    }
       edge from parent[<-]
       node[xshift=-15pt,midway,fill=white,color=white,text=black]{$\{\mathit{MouseMove},\mathit{MouseUp}\}$}
       %node[left]{$\{\mathit{MouseMove},$ \\ $\mathit{MouseUp}\}$}
  }
   child {node[bag] {|>>|}  [sibling distance=2.4cm]
child {node[bag] {|mouseDown|}
       edge from parent[<-]
       node[midway,fill=white,color=white,text=black]{$\{\mathit{MouseDown}\}$}
    }
    child {node[bag] {|deltaTime|}
       edge from parent[-]
    }
       edge from parent[<-]
       node[xshift=+15pt,midway,fill=white,color=white,text=black]{$\{\mathit{MouseDown}\}$}
  }
 
 ;
\node[interpret,above of=root,yshift=+20pt,name=inter,text width=3cm] {Reactive Interpreter};
\draw [->] (root) --  (inter) node[midway,fill=white,color=white,text=black]{$\{\mathit{MouseMove},\mathit{MouseUp}, \mathit{MouseDown}\}$};
\end{tikzpicture} 
\begin{code}
first (first mouseMove mouseUp) (mouseDown >> deltaTime) 
\end{code}
\end{frame}


\begin{frame}{Example evaluation}

\begin{tikzpicture}[grow=down, level distance=60pt]
[-]

\node[bag, name=root] {|first|}
  [sibling distance=4.5cm]
  child {node[bag] {|first|}[sibling distance=2cm]
    child {node[bag] {|mouseMove|}
       edge from parent[-,dashed]
       node[xshift=-15pt,midway,fill=white,color=white,text=black]{$\{\mathit{MouseMove}\}$}
    }
    child {node[bag] {|mouseUp|}
       edge from parent[-,dashed]
       node[xshift=+15pt,midway,fill=white,color=white,text=black]{$\{\mathit{MouseUp}\}$}
    }
       edge from parent[-,dashed]
       node[xshift=-15pt,midway,fill=white,color=white,text=black]{$\{\mathit{MouseMove},\mathit{MouseUp}\}$}
       %node[left]{$\{\mathit{MouseMove},$ \\ $\mathit{MouseUp}\}$}
  }
   child {node[bag] {|>>|}  [sibling distance=2.4cm]
child {node[bag] {|mouseDown|}
       edge from parent[->,very thick]
       node[midway,fill=white,color=white,text=black]{$\{\mathit{MouseDown}\}$}
    }
    child {node[bag] {|deltaTime|}
       edge from parent[-,dashed]
    }
       edge from parent[->,very thick]
       node[xshift=+15pt,midway,fill=white,color=white,text=black]{$\{\mathit{MouseDown}\}$}
  }
 
 ;
\node[interpret,above of=root,yshift=+20pt,name=inter,text width=3cm] {Reactive Interpreter};
\draw [<-,very thick] (root) --  (inter) node[midway,fill=white,color=white,text=black]{Observed: $\mathit{MouseDown}$};

\end{tikzpicture} 
\begin{code}
first (first mouseMove mouseUp) (mouseDown >> deltaTime) 
\end{code}
\end{frame}

\begin{frame}{Example evaluation}

\begin{tikzpicture}[grow=down, level distance=60pt]
[-]

\node[bag, name=root] {|first|}
  [sibling distance=4.5cm]
  child {node[bag] {|first|}[sibling distance=2cm]
    child {node[bag] {|mouseMove|}
       edge from parent[<-]
       node[xshift=-15pt,midway,fill=white,color=white,text=black]{$\{\mathit{MouseMove}\}$}
    }
    child {node[bag] {|mouseUp|}
       edge from parent[<-]
       node[xshift=+15pt,midway,fill=white,color=white,text=black]{$\{\mathit{MouseUp}\}$}
    }
       edge from parent[<-]
       node[xshift=-15pt,midway,fill=white,color=white,text=black]{$\{\mathit{MouseMove},\mathit{MouseUp}\}$}
       %node[left]{$\{\mathit{MouseMove},$ \\ $\mathit{MouseUp}\}$}
  }
   child {node[bag] {|deltaTime|}  [sibling distance=2.4cm]
       edge from parent[<-]
       node[xshift=+15pt,midway,fill=white,color=white,text=black]{$\{\mathit{DeltaTime}\}$}
  }
 
 ;
\node[interpret,above of=root,yshift=+20pt,name=inter,text width=3cm] {Reactive Interpreter};
\draw [->] (root) --  (inter) node[midway,fill=white,color=white,text=black]{$\{\mathit{MouseMove},\mathit{MouseUp}, \mathit{DeltaTime}\}$};
\end{tikzpicture} 
\begin{code}
first (first mouseMove mouseUp) deltaTime
\end{code}
\end{frame}

\begin{frame}{Why purely functional FRP evaluation matters}
Reactive level immutability: time-branching

Example: Duplicate drawing in drawing program into two tabs

Different ``Futures''

Impossible with evaluation models that rely on side effects for efficiency.

\end{frame}

\end{document}
